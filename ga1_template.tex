\documentclass[11pt]{article}
\usepackage[margin=1in]{geometry}
\usepackage{hyperref}
\usepackage{graphicx}
\usepackage{amsmath}
\usepackage{listings}
\usepackage{color}
\usepackage{fancyhdr}

% Code style settings
\definecolor{codegreen}{rgb}{0,0.6,0}
\definecolor{codegray}{rgb}{0.5,0.5,0.5}
\definecolor{codepurple}{rgb}{0.58,0,0.82}
\definecolor{backcolour}{rgb}{0.95,0.95,0.92}

\lstdefinestyle{mystyle}{
    backgroundcolor=\color{backcolour},   
    commentstyle=\color{codegreen},
    keywordstyle=\color{magenta},
    numberstyle=\tiny\color{codegray},
    stringstyle=\color{codepurple},
    basicstyle=\ttfamily\footnotesize,
    breakatwhitespace=false,         
    breaklines=true,                 
    captionpos=b,                    
    keepspaces=true,                 
    numbers=left,                    
    numbersep=5pt,                  
    showspaces=false,                
    showstringspaces=false,
    showtabs=false,                  
    tabsize=2
}

\lstset{style=mystyle}

% Header/footer
\pagestyle{fancy}
\fancyhf{}
\rhead{CS 312: Group Assignment \#1}
\lfoot{Group \#} % TODO: Add your group number
\rfoot{Page \thepage}

\begin{document}

% Title page
\begin{center}
\Large{\textbf{CS 312: Group Assignment \#1}}\\
\large{\textbf{Ethical Analysis and Bias Investigation in Large Language Models}}\\
\vspace{1cm}
\normalsize{\textbf{Group \#}} % TODO: Add your group number\\
\vspace{0.5cm}
\normalsize{
% TODO: Add all group member names and IDs (one per line)
Name (ID: )\\
Name (ID: )\\
Name (ID: )\\
}\\
\vspace{0.5cm}
\normalsize{Submitted: \today}
\end{center}

\vspace{1cm}

% ============================================
\section{Part 1: Model Setup and Ethical Framework}
% ============================================

\subsection{Setting Up Gemma}

% TODO: Include screenshot of successful model loading
% \begin{figure}[h]
%     \centering
%     \includegraphics[width=0.8\textwidth]{filename.png}
%     \caption{Your caption here}
%     \label{fig:model_setup}
% \end{figure}

\textbf{Model Specifications:}
\begin{itemize}
    \item \textbf{Vocabulary Size:} % TODO: Add vocabulary size
    \item \textbf{Parameter Count:} % TODO: Add parameter count
\end{itemize}

\textbf{Explanation:}

% TODO: Explain what these numbers mean for the model's capabilities (100-150 words)

\subsection{Ethical Issue Selection and Framework Application}

\textbf{Selected Ethical Issue:} % TODO: State your chosen issue

% TODO: Write 2-3 paragraphs (300-500 words) addressing:
% 1. Definition of the issue with 2-3 concrete examples
% 2. Application of your chosen ethical framework (Utilitarianism or Virtue Ethics)  
% 3. Stakeholder analysis with at least 3 distinct groups

\subsection{Current Event Analysis}

% TODO: Write approximately 500 words analyzing a real 2024-2025 event
% Must include:
% 1. Event description with proper citations
% 2. Concrete harm identification
% 3. Connection to your ethical issue
% 4. Application of ethical framework
% 5. Discussion of implications

% ============================================
\section{Part 2: Technical Investigation}
% ============================================

\subsection{Hypothesis Development and Test Design}

\textbf{Hypotheses:}
\begin{enumerate}
    \item % TODO: Add hypothesis 1 (specific and measurable)
    \item % TODO: Add hypothesis 2 (specific and measurable)
    \item % TODO: Add hypothesis 3 (specific and measurable)
    % TODO: Add more hypotheses as needed (3-5 total)
\end{enumerate}

\textbf{Test Categories and Measurement:}

\begin{center}
\begin{tabular}{|p{3cm}|p{4cm}|p{4cm}|}
\hline
\textbf{Category} & \textbf{What it Tests} & \textbf{Measurement Method} \\
\hline
% TODO: Add rows for each test category
 &  &  \\
\hline
 &  &  \\
\hline
\end{tabular}
\end{center}

\textbf{Evaluation Criteria:}

% TODO: Describe your evaluation criteria with specific examples

\subsection{Prompt Engineering and Testing}

\textbf{Summary of Testing Approach:}

% TODO: Describe your implementation approach

% TODO: Include relevant code snippets
\begin{lstlisting}[language=Python, caption=Your Caption]
# Your code here
\end{lstlisting}

\textbf{Testing Methodology:}
\begin{itemize}
    \item \textbf{Number of base prompts:} % TODO: Add number
    \item \textbf{Variables tested:} % TODO: List variables
    \item \textbf{Total test cases:} % TODO: Add total
    \item \textbf{Issues encountered:} % TODO: Describe any issues
\end{itemize}

\textit{Note: Full code implementation available in \texttt{group\_\#\_ga1.ipynb}}

\subsection{Results Analysis and Visualization}

\textbf{Quantitative Results:}

% TODO: Add your statistics and metrics

% TODO: Include visualizations
% \begin{figure}[h]
%     \centering
%     \includegraphics[width=0.7\textwidth]{filename.png}
%     \caption{Your caption}
%     \label{fig:results}
% \end{figure}

\textbf{Qualitative Analysis:} 

% TODO: Write 300-400 words discussing:
% - Which prompts most reliably surfaced the issue
% - Surprising patterns or edge cases
% - Support/contradiction of hypotheses
% - Specific example responses

\textbf{Pattern Identification:}

% TODO: Describe systematic patterns, triggering factors, and consistency

% ============================================
\section{Part 3: Intervention Development and Evaluation}
% ============================================

\subsection{Intervention Design}

\textbf{Intervention Approach:}

% TODO: Write 400-500 words describing:
% 1. Your intervention approach (system prompts, prompt engineering, etc.)
% 2. Theoretical basis for why this should work
% 3. Connection to findings from Part 2

\textbf{Justification:}

% TODO: Explain why this approach is better than alternatives, limitations, edge cases

\textbf{Implementation Plan:}

% TODO: Provide step-by-step description with pseudocode or flowchart

\subsection{Intervention Implementation and Testing}

\textbf{Implementation Summary:}

% TODO: Describe how you implemented the intervention

\textbf{Quantitative Results:}
\begin{itemize}
    \item % TODO: Add improvement metrics
\end{itemize}

% TODO: Include before/after visualizations
% \begin{figure}[h]
%     \centering
%     \includegraphics[width=0.8\textwidth]{filename.png}
%     \caption{Your caption}
%     \label{fig:intervention}
% \end{figure}

\textbf{Analysis of Effectiveness:}

% TODO: Write 300-400 words discussing:
% - Quantitative improvement (or lack thereof)
% - Categories where intervention worked best/worst
% - Unexpected behaviors or side effects
% - Specific examples comparing responses

\textbf{Edge Case Testing:}

% TODO: Describe results from adversarial prompt testing

\subsection{Research Implications and Future Directions}

% TODO: Write 600-800 words total covering:

\textbf{Generalizability:}

% Discuss applicability to other LLMs, ethical issues, and variation factors

\textbf{Real-world Deployment:}

% Discuss scale implementation, costs, and responsibility

\textbf{Research Extensions:}

% Discuss what a full research paper would include, additional experiments, open questions

\textbf{Policy Implications:}

% Discuss governance, regulations, and stakeholder roles

% ============================================
\section*{References}
% ============================================

% TODO: Add all your citations in proper academic format

% ============================================
\section*{Appendix: AI Tool Usage}
% ============================================

% TODO: Document any AI tool usage as required by the assignment
% If no AI tools used, state: "No AI tools were used in completing this assignment beyond those explicitly allowed."

\end{document}